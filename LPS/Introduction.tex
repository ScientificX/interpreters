\chapter{Introduction}

LPS is a Language Prototyping System based on 
Modular Monadic Semantics.

For more information, see: 

http://lsi.uniovi.es/~labra/LPS/LPS.html

Some papers describing the system are \cite{Labra98, Labra99}.


Modular Monadic Semantics (\cite{LiangHudakJones95, LiangHudak96}) 
separates values from computations. 

A computation can be described by means of a monad \cite{Moggi89}. Using monad 
transformers, a monad can be transformed into a different monad with
more computational features.

The system can be seen as a Domain Specific Language embedded in
Haskell \cite{Haskell98}.

In order to derive a programming language specification it is
necessary to provide:

\begin{itemize}
\item A \emph{Parser} for the language
\item A description of the domain values 
\item A characterization of the monad that models the 
	computation
\end{itemize}


