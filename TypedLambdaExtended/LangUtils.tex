\documentclass[11pt]{article}

%% To convert to tex format use lhs2TeX --poly filename.lhs > filename.tex

%% ODER: format ==         = "\mathrel{==}"
%% ODER: format /=         = "\neq "
%
%
\makeatletter
\@ifundefined{lhs2tex.lhs2tex.sty.read}%
  {\@namedef{lhs2tex.lhs2tex.sty.read}{}%
   \newcommand\SkipToFmtEnd{}%
   \newcommand\EndFmtInput{}%
   \long\def\SkipToFmtEnd#1\EndFmtInput{}%
  }\SkipToFmtEnd

\newcommand\ReadOnlyOnce[1]{\@ifundefined{#1}{\@namedef{#1}{}}\SkipToFmtEnd}
\usepackage{amstext}
\usepackage{amssymb}
\usepackage{stmaryrd}
\DeclareFontFamily{OT1}{cmtex}{}
\DeclareFontShape{OT1}{cmtex}{m}{n}
  {<5><6><7><8>cmtex8
   <9>cmtex9
   <10><10.95><12><14.4><17.28><20.74><24.88>cmtex10}{}
\DeclareFontShape{OT1}{cmtex}{m}{it}
  {<-> ssub * cmtt/m/it}{}
\newcommand{\texfamily}{\fontfamily{cmtex}\selectfont}
\DeclareFontShape{OT1}{cmtt}{bx}{n}
  {<5><6><7><8>cmtt8
   <9>cmbtt9
   <10><10.95><12><14.4><17.28><20.74><24.88>cmbtt10}{}
\DeclareFontShape{OT1}{cmtex}{bx}{n}
  {<-> ssub * cmtt/bx/n}{}
\newcommand{\tex}[1]{\text{\texfamily#1}}	% NEU

\newcommand{\Sp}{\hskip.33334em\relax}

\newlength{\lwidth}\setlength{\lwidth}{4.5cm}
\newlength{\cwidth}\setlength{\cwidth}{8mm} % 3mm

\newcommand{\Conid}[1]{\mathit{#1}}
\newcommand{\Varid}[1]{\mathit{#1}}
\newcommand{\anonymous}{\kern0.06em \vbox{\hrule\@width.5em}}
\newcommand{\plus}{\mathbin{+\!\!\!+}}
\newcommand{\bind}{\mathbin{>\!\!\!>\mkern-6.7mu=}}
\newcommand{\sequ}{\mathbin{>\!\!\!>}}
\renewcommand{\leq}{\leqslant}
\renewcommand{\geq}{\geqslant}
\newcommand{\NB}{\textbf{NB}}
\newcommand{\Todo}[1]{$\langle$\textbf{To do:}~#1$\rangle$}

\makeatother
\EndFmtInput
%

\usepackage{alltt}

\bibliographystyle{plain}

\parskip=\medskipamount
\parindent=0pt

\title{Algebraic Language Utilities}
\author{\emph{Uk'taad B'mal} \\
        The University of Kansas - ITTC \\
        2335 Irving Hill Rd, Lawrence, KS 66045 \\
        \texttt{lambda@ittc.ku.edu}}

\begin{document}

\maketitle

\begin{abstract}
  This literate script provides a collection of algebraic structures
  useful for defining languages and types.  These idea are drawn from
  several sources in the literature and simply represent a documented,
  standard implementation for use in language definitions.
  Documentation is intended to be sufficient for those learning to use
  the algebraic structures in language processor design.
\end{abstract}

\section{Introduction}

The \ensuremath{\Conid{LangUtils}} module provides classes, data structures and functions
for developing languages and language transformations for
compositional language definition.  The \ensuremath{\Conid{Sum}} type constructor is used
to compose languages defined over value spaces.  The \ensuremath{\Conid{Functor}} and
\ensuremath{\Conid{Algebra}} constructors are used to provide transformation and
interpretation functions for language elements.  The \ensuremath{\Varid{cata}} function
defines a catamorphism that combines transformation and interpretation
functions and defines evaluation function for languages.  The \ensuremath{\Conid{Subsum}}
provides mechanisms for asserting subtype relationship between
language elements and the main language.

\begin{tabbing}
\qquad\=\hspace{\lwidth}\=\hspace{\cwidth}\=\+\kill
${\mbox{\enskip\{-\# OPTIONS -fglasgow-exts -fallow-undecidable-instances -fallow-overlapping-instances  \#-\}\enskip}}$\\
${\mathbf{module}\;\Conid{LangUtils}\;\mathbf{where}}$\\
${}$\\
${\mathbf{import}\;\Conid{\Conid{Control}.Monad}}$\\
${\mathbf{import}\;\Conid{\Conid{Control}.\Conid{Monad}.Reader}}$\\
${\mathbf{import}\;\Conid{\Conid{Control}.\Conid{Monad}.Error}}$\\
${\mathbf{import}\;\Conid{\Conid{Data}.Maybe}}$\\
${\mathbf{import}\;\Conid{\Conid{Data}.List}}$
\end{tabbing}
\section{The Sum Type}

First define the \ensuremath{\Conid{Sum}} data type that will allow us to combine language
elements.  \ensuremath{\Varid{f}} and \ensuremath{\Varid{g}} are language elements and \ensuremath{\Varid{x}} is the value space
over which the language elements are defined:

\begin{tabbing}
\qquad\=\hspace{\lwidth}\=\hspace{\cwidth}\=\+\kill
${\mathbf{data}\;\Conid{Sum}\;\Varid{f}\;\Varid{g}\;\Varid{x}\mathrel{=}\Conid{S}\;(\Conid{Either}\;(\Varid{f}\;\Varid{x})\;(\Varid{g}\;\Varid{x}))}$\\
${\Varid{unS}\;(\Conid{S}\;\Varid{x})\mathrel{=}\Varid{x}}$
\end{tabbing}
Note that the \ensuremath{\Conid{Sum}} type differs from the standard \ensuremath{\Conid{Either}} type with
the presence of a third parameter.

\section{Functors and Algebras}

To understand the \ensuremath{\Conid{Functor}} and \ensuremath{\Conid{Algebra}} classes with respect to
languages it is important to remember that in our definitions \ensuremath{\Varid{a}} is a
value space and \ensuremath{\Varid{f}\;\Varid{a}} is a term language defined over that value
space.  For example, if \ensuremath{\Varid{a}} is \ensuremath{\Conid{Integer}} and \ensuremath{\Varid{f}} is \ensuremath{\Conid{List}}, then \ensuremath{\Varid{f}\;\Varid{a}}
is the language of lists defined over integers.  When we define
functors, we are defining a transform, \ensuremath{\Varid{map}_{f}}, from a language defined
over one value space to the same language defined over another value
space.  When we define algebras, we are defining a transform, \ensuremath{\Varid{\phi}},
from a language defined over a value space to the value space itself.
In effect, \ensuremath{\Varid{\phi}} provides an evaluation function.

It is important to remember the definition of \ensuremath{\Conid{Functor}} provide in the
standard Prelude:

\begin{tabbing}
\qquad\=\hspace{\lwidth}\=\hspace{\cwidth}\=\+\kill
${\mathbf{class}\;\Conid{Functor}\;\Varid{f}\;\mathbf{where}}$\\
${\hskip2.00em\relax\Varid{map}_{f}\mathbin{::}(\Varid{a}\to \Varid{b})\to \Varid{f}\;\Varid{a}\to \Varid{f}\;\Varid{b}}$
\end{tabbing}
A \ensuremath{\Conid{Functor}} is some type constructor, \ensuremath{\Varid{f}} that we can define a map,
\ensuremath{\Varid{map}_{f}}, over.  The signature of \ensuremath{\Varid{map}_{f}} states that given a function
mapping type \ensuremath{\Varid{a}} to type \ensuremath{\Varid{b}} and an instance of the functor \ensuremath{\Varid{f}\;\Varid{a}},
\ensuremath{\Varid{map}_{f}} will generate an instance of the functor over \ensuremath{\Varid{b}}, \ensuremath{\Varid{f}\;\Varid{b}}.  For
example, if \ensuremath{\Varid{f}} is a \ensuremath{\Conid{List}} type constructor, then \ensuremath{\Varid{map}_{f}} is the built
in \ensuremath{\Varid{map}} function.  In language processing, \ensuremath{\Varid{f}\;\Varid{a}} is an abstract
syntax tree representing a language and \ensuremath{\Varid{map}_{f}} is a set of
transformations over language elements.  If \ensuremath{\Varid{a}\equiv \Varid{b}} then we are
defining language transformations within the same language.

We define \ensuremath{\Conid{Algebra}} and \ensuremath{\Conid{Coalgebra}} classes that define \ensuremath{\Varid{\phi}} and
\ensuremath{\Varid{\psi}} respectively.  If \ensuremath{\Varid{f}} is a functor, then \ensuremath{\Varid{f}\;\Varid{a}} is an \ensuremath{\Conid{Algebra}}
if \ensuremath{\Varid{\phi}} can be defined mapping an application of \ensuremath{\Varid{f}} to it's argument
type \ensuremath{\Varid{a}}.  Conversely, the \ensuremath{\Conid{Coalgebra}} defines a mapping \ensuremath{\Varid{\psi}} from
the argument type to a \ensuremath{\Conid{Functor}} application.

\begin{tabbing}
\qquad\=\hspace{\lwidth}\=\hspace{\cwidth}\=\+\kill
${\mathbf{class}\;\Conid{Functor}\;\Varid{f}\Rightarrow \Conid{Algebra}\;\Varid{f}\;\Varid{a}\;\mathbf{where}}$\\
${\hskip2.00em\relax\Varid{\phi}\mathbin{::}\Varid{f}\;\Varid{a}\to \Varid{a}}$\\
${}$\\
${\mathbf{class}\;\Conid{Functor}\;\Varid{f}\Rightarrow \Conid{Coalgebra}\;\Varid{f}\;\Varid{a}\;\mathbf{where}}$\\
${\hskip2.00em\relax\Varid{\psi}\mathbin{::}\Varid{a}\to \Varid{f}\;\Varid{a}}$
\end{tabbing}
From an algebraic perspective, \ensuremath{\Varid{a}} is the carrier set over which the
algebra \ensuremath{\Varid{f}} is defined.  \ensuremath{\Varid{f}} defines the terms of the algebra.  \ensuremath{\Varid{\phi}}
is the mapping from terms to the carrier set.  Thus, \ensuremath{\Varid{\phi}} defines the
value from the carrier set associated with terms in \ensuremath{\Varid{f}}.  That is why we
use \ensuremath{\Varid{\phi}} for evaluation when defining languages.

\section{Fixed Point Types}

The \ensuremath{\Conid{Rec}} type constructor defines a fixed point type for some type
\ensuremath{\Varid{f}} having one type argument.  The fixed point type creates a
parameter-less type from the single parameter type that represents all
possible individual terms in the language by instantiating the carrier
set parameter with the language itself.  The constructor \ensuremath{\Conid{In}} is
necessary \texttt{Haskell} machinery for defining a data type.  \ensuremath{\Varid{out}}
is a utility function for unpackaging the fixed point from the type
constructor.

\begin{tabbing}
\qquad\=\hspace{\lwidth}\=\hspace{\cwidth}\=\+\kill
${\mathbf{data}\;\Conid{Rec}\;\Varid{f}\mathrel{=}\Conid{In}\;(\Varid{f}\;(\Conid{Rec}\;\Varid{f}))}$\\
${\Varid{out}\mathbin{::}\Conid{Rec}\;\Varid{f}\to \Varid{f}\;(\Conid{Rec}\;\Varid{f})}$\\
${\Varid{out}\;(\Conid{In}\;\Varid{x})\mathrel{=}\Varid{x}}$
\end{tabbing}
To understand fixed point types and what they achieve, consider the
following data type definitions:

\begin{tabbing}
\qquad\=\hspace{\lwidth}\=\hspace{\cwidth}\=\+\kill
${\mathbf{data}\;\Conid{Expr}\;\Varid{e}}$\\
${\hskip2.00em\relax\mathrel{=}\Conid{Val}\;\Conid{Int}}$\\
${\hskip2.00em\relax\mid \Conid{Add}\;\Varid{e}\;\Varid{e}}$\\
${}$\\
${\mathbf{type}\;\Conid{Lang0}\mathrel{=}\Conid{Expr}\;\Conid{Int}}$
\end{tabbing}
The elements of \ensuremath{\Conid{Lang0}} are things that can be constructed using \ensuremath{\Conid{Val}}
and \ensuremath{\Conid{Add}} over \ensuremath{\Conid{Int}}.  Thus, \ensuremath{(\Conid{Add}\;\mathrm{1}\;\mathrm{2})} and \ensuremath{(\Conid{Val}\;\mathrm{3})} are correct
elements of the language.  However, \ensuremath{(\Conid{Add}\;(\Conid{Val}\;\mathrm{2})\;(\Conid{Val}\;\mathrm{4}))} and \ensuremath{(\Conid{Add}\;(\Conid{Add}\;\mathrm{1}\;\mathrm{2})\;\mathrm{3})} are not because the arguments to \ensuremath{\Conid{Add}} are not of the
type used to isntantiate \ensuremath{\Varid{e}},specifically \ensuremath{\Conid{Int}}.

The first fix to this problem would be to instantate \ensuremath{\Varid{e}} with \ensuremath{\Conid{Expr}\;\Conid{Int}} as follows:

\begin{tabbing}
\qquad\=\hspace{\lwidth}\=\hspace{\cwidth}\=\+\kill
${\mathbf{type}\;\Conid{Lang1}\mathrel{=}\Conid{Expr}\;(\Conid{Expr}\;\Conid{Int})}$
\end{tabbing}
Now the arguments to constructors can be of type \ensuremath{\Conid{Expr}\;\Conid{Int}}.  Now the
constructions \ensuremath{(\Conid{Add}\;(\Conid{Val}\;\mathrm{1})\;(\Conid{Val}\;\mathrm{2}))} and \ensuremath{(\Conid{Add}\;(\Conid{Add}\;\mathrm{1}\;\mathrm{2})\;(\Conid{Const}\;\mathrm{3}))}
are legal language elements.  Unfortunately, we have only defered the
problem.  The construction \ensuremath{\Conid{Add}\;(\Conid{Add}\;(\Conid{Add}\;\mathrm{1}\;\mathrm{2})\;(\Conid{Add}\;\mathrm{2}\;\mathrm{3}))} is still
illegal.

It should be clear that defining:

\begin{tabbing}
\qquad\=\hspace{\lwidth}\=\hspace{\cwidth}\=\+\kill
${\mathbf{type}\;\Conid{Lang2}\mathrel{=}\Conid{Expr}\;(\Conid{Expr}\;(\Conid{Expr}\;\Conid{Int}))}$
\end{tabbing}
simply defers the problem further.

The problem is that the language we want where \ensuremath{\Conid{Add}} can be nested
arbitrarily is an infinite recursion.  No matter how \ensuremath{\Conid{Expr}} is nested
above, the nesting will be finite.  In this language, there will
always be an \ensuremath{\Conid{Add}} instance whose arguments must be of type \ensuremath{\Conid{Int}}.

The fixed point constructor solves this problem by lazily finding the
fixed point of langauge.  What if we specified the following (ignoring
Haskell details for the moment):

\begin{tabbing}
\qquad\=\hspace{\lwidth}\=\hspace{\cwidth}\=\+\kill
${\mathbf{type}\;\Conid{Lang3}\mathrel{=}\Conid{Expr}\;\Conid{Lang3}}$
\end{tabbing}
In this interesting recursive structure, the parameter to \ensuremath{\Conid{Expr}} to
create \ensuremath{\Conid{Lang3}} is \ensuremath{\Conid{Lang3}} itself.  Thus, an \ensuremath{\Conid{Add}} can be instantiated
with any element of \ensuremath{\Varid{lang3}} or any \ensuremath{\Conid{Expr}}.  The reason for the \ensuremath{\Conid{Val}}
constructor should nobe apparent because something must stop the
recursive construction.  Because \ensuremath{\Conid{Val}} does not depend on the \ensuremath{\Varid{e}}
parameter, \ensuremath{\Conid{Expr}} values cannot be nested in \ensuremath{\Conid{Val}}.

Unfortunately, the semantics of \ensuremath{\mathbf{type}} in \texttt{Haskell} do not
allow the construction above.  A datatype must be used requiring the
inclusion of a constructor.  Thus, a constructor for the data type
must be defined to keep \texttt{Haskell} happy resulting in the
generalized definition:

\begin{tabbing}
\qquad\=\hspace{\lwidth}\=\hspace{\cwidth}\=\+\kill
${\mathbf{data}\;\Conid{Rec}\;\Conid{F}\mathrel{=}\Conid{In}\;(\Conid{F}\;(\Conid{Rec}\;\Conid{F}))}$
\end{tabbing}
\section{Catamorphism}

The \ensuremath{\Varid{cata}} function represents a catamorphism, or fold, over a fixed
point recursive structure. The constraint \ensuremath{\Conid{Algebra}\;\Varid{f}\;\Varid{a}} assures that
\ensuremath{\Varid{f}} is an algebra and transitively that \ensuremath{\Varid{f}} is a functor.  Thus, both
\ensuremath{\Varid{\phi}} and \ensuremath{\Varid{map}_{f}} exist with respect to \ensuremath{\Varid{f}\;\Varid{a}}.  \ensuremath{\Varid{cata}} is the
composition of three functions, \ensuremath{\Varid{\phi}}, \ensuremath{\Varid{map}_{f}\;\Varid{cata}} and \ensuremath{\Varid{out}}.  \ensuremath{\Varid{out}}
takes a recursive type and removes the \ensuremath{\Conid{In}} type constructor
introduced by \ensuremath{\Conid{Rec}}.  \ensuremath{\Varid{map}_{f}\;\Varid{cata}} maps the \ensuremath{\Varid{cata}} function over \ensuremath{\Varid{f}}
effectively pushing the \ensuremath{\Varid{cata}} operation into \ensuremath{\Varid{f}}.  This effectively
evaluates the subterms of any term currently being evaluated.
Finally, \ensuremath{\Varid{\phi}} evaluates the result of \ensuremath{\Varid{map}_{f}\;\Varid{cata}} by transforming the
result of the catamorphism to a value in the carrier set. From a
language perspective, \ensuremath{\Varid{map}_{f}\;\Varid{cata}} pushes the evaluation into the
sub-expression of the argument expression while \ensuremath{\Varid{\phi}} evaluates the
result to an element of the value space.

\begin{tabbing}
\qquad\=\hspace{\lwidth}\=\hspace{\cwidth}\=\+\kill
${\Varid{cata}\mathbin{::}(\Conid{Algebra}\;\Varid{f}\;\Varid{a})\Rightarrow \Conid{Rec}\;\Varid{f}\to \Varid{a}}$\\
${\Varid{cata}\mathrel{=}\Varid{\phi}\mathbin{\circ}\Varid{map}_{f}\;\Varid{cata}\mathbin{\circ}\Varid{out}}$
\end{tabbing}
An interesting property of \ensuremath{\Varid{cata}} is that it provides an evaluation
capability for \emph{any} language described by an \ensuremath{\Conid{Algebra}} of the
form \ensuremath{\Varid{f}\;\Varid{a}}.  Thus, as long as the \ensuremath{\Conid{Algebra}} property is established,
we automatically get \ensuremath{\Varid{cata}} for our evaluation function.  As we shall
see later, it is useful to actually define \ensuremath{\Varid{eval}} with a signature
that directly constrains the types associated with \ensuremath{\Varid{cata}}.

\section{Combining Language Elements}

Earlier the \ensuremath{\Conid{Sum}} type was defined to combine data types.  If we wish
to combine language elements represented as data types using the
\ensuremath{\Conid{Sum}}, then the resulting structure must also be a functor and \ensuremath{\Varid{map}_{f}}
must be defined over the sum. This is quite easily done using the
\ensuremath{\Conid{Either}} type encapsulated in the \ensuremath{\Conid{Sum}}.  If \ensuremath{\Varid{f}} and \ensuremath{\Varid{g}} are functors,
then \ensuremath{\Conid{Sum}\;\Varid{f}\;\Varid{g}} is a functor where \ensuremath{\Varid{map}_{f}} selects the \ensuremath{\Varid{map}_{f}} associated
with \ensuremath{\Varid{f}} or \ensuremath{\Varid{g}} depending on whether it is operating on the left or
right part of the sum.

\begin{tabbing}
\qquad\=\hspace{\lwidth}\=\hspace{\cwidth}\=\+\kill
${\mathbf{instance}\;(\Conid{Functor}\;\Varid{f},\Conid{Functor}\;\Varid{g})\Rightarrow \Conid{Functor}\;(\Conid{Sum}\;\Varid{f}\;\Varid{g})\;\mathbf{where}}$\\
${\hskip2.00em\relax\Varid{map}_{f}\;\Varid{h}\;(\Conid{S}\;(\Conid{\Conid{Prelude}.Left}\;\Varid{x}))\mathrel{=}\Conid{S}\;(\Conid{\Conid{Prelude}.Left}\;(\Varid{map}_{f}\;\Varid{h}\;\Varid{x}))}$\\
${\hskip2.00em\relax\Varid{map}_{f}\;\Varid{h}\;(\Conid{S}\;(\Conid{\Conid{Prelude}.Right}\;\Varid{x}))\mathrel{=}\Conid{S}\;(\Conid{\Conid{Prelude}.Right}\;(\Varid{map}_{f}\;\Varid{h}\;\Varid{x}))}$
\end{tabbing}
If \ensuremath{\Varid{f}\;\Varid{a}} and \ensuremath{\Varid{g}\;\Varid{a}} are algebras, then \ensuremath{\Conid{Sum}\;\Varid{f}\;\Varid{g}} is also an algebra
were \ensuremath{\Varid{\phi}} is the sum of the \ensuremath{\Varid{\phi}} functions from the original
algebras.  Here we use the built-in \ensuremath{\Varid{either}} function to select the
appropriate \ensuremath{\Varid{\phi}}.  \ensuremath{\Varid{unS}} is composed with \ensuremath{\Varid{either}\;\Varid{\phi}\;\Varid{\phi}} to remove
the \ensuremath{\Conid{S}} constructor introduced by the \ensuremath{\Conid{Sum}}.

\begin{tabbing}
\qquad\=\hspace{\lwidth}\=\hspace{\cwidth}\=\+\kill
${\mathbf{instance}\;(\Conid{Algebra}\;\Varid{f}\;\Varid{a},\Conid{Algebra}\;\Varid{g}\;\Varid{a})\Rightarrow \Conid{Algebra}\;(\Conid{Sum}\;\Varid{f}\;\Varid{g})\;\Varid{a}}$\\
${\hskip2.00em\relax\mathbf{where}\;\Varid{\phi}\mathrel{=}\Varid{either}\;\Varid{\phi}\;\Varid{\phi}\mathbin{\circ}\Varid{unS}}$
\end{tabbing}
With the definition of \ensuremath{\Conid{Functor}} and \ensuremath{\Conid{Algebra}} over the \ensuremath{\Conid{Sum}}
constructor, any \ensuremath{\Conid{Algebra}}s composed using the \ensuremath{\Conid{Sum}} type are
themselves \ensuremath{\Conid{Algebra}}s.  The implication to languages is we can define
individual language elements and compose them into larger languages
using the \ensuremath{\Conid{Sum}} type.  Furthermore, we know the resulting language is
an instance of \ensuremath{\Conid{Algebra}} and \ensuremath{\Conid{Functor}} making \ensuremath{\Varid{cata}} available as an
evaluation function.

\section{Subtypes and Subsums}

We define \ensuremath{\Conid{Subtype}} over two types in the classical way by defining
\ensuremath{\Varid{\uparrow}} (injection) and \ensuremath{\Varid{\downarrow}} (projection) functions between the types.
If \ensuremath{\Varid{a}} is a subtype of \ensuremath{\Varid{b}}, then it will always be possible to inject
an element of \ensuremath{\Varid{a}} into the type \ensuremath{\Varid{b}}.  This is not the case for the
projection function, thus we use the \ensuremath{\Conid{Maybe}} type for the domain of
\ensuremath{\Varid{\downarrow}}.

\begin{tabbing}
\qquad\=\hspace{\lwidth}\=\hspace{\cwidth}\=\+\kill
${\mathbf{class}\;\Conid{Subtype}\;\Varid{a}\;\Varid{b}\;\mathbf{where}}$\\
${\hskip2.00em\relax\Varid{\uparrow}\mathbin{::}\Varid{a}\to \Varid{b}}$\\
${\hskip2.00em\relax\Varid{\downarrow}\mathbin{::}\Varid{b}\to \Conid{Maybe}\;\Varid{a}}$
\end{tabbing}
We can now define some standard subtype relationships using the
\ensuremath{\Conid{Subtype}} constructor.  First, every type \ensuremath{\Varid{x}} is a subtype of itself
where \ensuremath{\Varid{\uparrow}\mathrel{=}\Varid{id}} and \ensuremath{\Varid{\downarrow}\mathrel{=}\Conid{Just}}.  When injecting a member of a type
into itself, we simply want the element.  Because we can always
project a member of a type into itself, we simply use \ensuremath{\Conid{Just}} to
encapsulate the type element.

\begin{tabbing}
\qquad\=\hspace{\lwidth}\=\hspace{\cwidth}\=\+\kill
${\mathbf{instance}\;\Conid{Subtype}\;\Varid{x}\;\Varid{x}\;\mathbf{where}}$\\
${\hskip2.00em\relax\Varid{\uparrow}\mathrel{=}\Varid{id}}$\\
${\hskip2.00em\relax\Varid{\downarrow}\mathrel{=}\Conid{Just}}$
\end{tabbing}
To make things easier in a traditional interpreter structure, we
define a projection function that uses the \ensuremath{\Conid{Error}} monad to throw an
error message in the traditional manner.  This function is technically
not required, but does make using the definitions simpler in real
interpreters.  Skip this function if you find it difficult to follow
as it is not used in the remainder of the paper.

\begin{tabbing}
\qquad\=\hspace{\lwidth}\=\hspace{\cwidth}\=\+\kill
${\Varid{mPrj}\mathbin{::}(\Conid{Error}\;\Varid{e},(\Conid{MonadError}\;\Varid{e}\;\Varid{m}),\Conid{Show}\;\Varid{b},\Conid{Subtype}\;\Varid{a}\;\Varid{b})\Rightarrow \Varid{b}\to \Varid{m}\;\Varid{a}}$\\
${\Varid{mPrj}\;\Varid{x}\mathrel{=}\Varid{maybe}\;(\Varid{throwError}\mathbin{\$}\Varid{strMsg}\;\Varid{errmsg})\;\Varid{return}\;(\Varid{\downarrow}\;\Varid{x})}$\\
${\hskip2.00em\relax\mathbf{where}}$\\
${\hskip2.00em\relax\Varid{errmsg}\mathrel{=}\text{\tt \char34 Type~error:~Cannot~project~`\char34}\plus }$\\
${\hskip2.00em\relax\phantom{\Varid{errmsg}\mathrel{=}\mbox{}}(\Varid{show}\;\Varid{x})\plus }$\\
${\hskip2.00em\relax\phantom{\Varid{errmsg}\mathrel{=}\mbox{}}\text{\tt \char34 '~to~the~desired~type\char34}}$\\
${}$\\
${\Varid{retInj}\;\Varid{x}\mathrel{=}\Varid{return}\;(\Varid{\uparrow}\;\Varid{x})}$
\end{tabbing}
\ensuremath{\Conid{Subsum}} is a version of \ensuremath{\Conid{Subtype}} where \ensuremath{\Varid{f}} and \ensuremath{\Varid{g}} are parameterized
 over a common type.  The reason for \ensuremath{\Conid{Subsum}}s' existance is that we
 would like to define languages as compositions of algebras using
 language elements.  All such language elements are parameterized over
 the value space or carrier set.  This is difficult to deal with using
 the traditional \ensuremath{\Conid{Subtype}} definition above.  The injection and
 projection functions are defined similarly using the \ensuremath{\Conid{Maybe}} type
 when defining projection.

\begin{tabbing}
\qquad\=\hspace{\lwidth}\=\hspace{\cwidth}\=\+\kill
${\mathbf{class}\;\Conid{Subsum}\;\Varid{f}\;\Varid{g}\;\mathbf{where}}$\\
${\hskip2.00em\relax\Varid{\uparrow}_{S}\mathbin{::}\Varid{f}\;\Varid{x}\to \Varid{g}\;\Varid{x}}$\\
${\hskip2.00em\relax\Varid{\downarrow}_{S}\mathbin{::}\Varid{g}\;\Varid{x}\to \Conid{Maybe}\;(\Varid{f}\;\Varid{x})}$
\end{tabbing}
As every \ensuremath{\Varid{f}} is a \ensuremath{\Conid{Subtype}} of itself, every \ensuremath{\Varid{f}} is a \ensuremath{\Conid{Subsum}} of
itself.  Again, the same logic is followed from the \ensuremath{\Conid{Subtype}}
definition.

\begin{tabbing}
\qquad\=\hspace{\lwidth}\=\hspace{\cwidth}\=\+\kill
${\mathbf{instance}\;\Conid{Subsum}\;\Varid{f}\;\Varid{f}\;\mathbf{where}}$\\
${\hskip2.00em\relax\Varid{\uparrow}_{S}\;\Varid{f}\mathrel{=}\Varid{f}}$\\
${\hskip2.00em\relax\Varid{\downarrow}_{S}\mathrel{=}\Conid{Just}}$
\end{tabbing}
We will prescribe how \ensuremath{\Conid{Sum}} is used to construct types to enable the
definition of injection and projection functions.  Specifically, we
will assume that \ensuremath{\Conid{Sum}} is used to compose types in the following
manner:

\begin{tabbing}
\qquad\=\hspace{\lwidth}\=\hspace{\cwidth}\=\+\kill
${\Varid{newType}\mathrel{=}(\Conid{Sum}\;\Varid{T}_{0}}$\\
${\phantom{\Varid{newType}\mathrel{=}(\mbox{}}(\Conid{Sum}\;\Varid{T}_{1}}$\\
${\phantom{\Varid{newType}\mathrel{=}(\mbox{}}\phantom{(\mbox{}}(\Conid{Sum}\;\Varid{T}_{2}}$\\
${\phantom{\Varid{newType}\mathrel{=}(\mbox{}}\phantom{(\mbox{}}\phantom{(\mbox{}}\mathbin{...}}$\\
${\phantom{\Varid{newType}\mathrel{=}(\mbox{}}\phantom{(\mbox{}}\phantom{(\mbox{}}\hskip0.50em\relax(\Conid{Sum}\;\Conid{Tm}\;\Conid{Tn})\mathbin{...})))}$
\end{tabbing}
Given this, we know that:

\begin{tabbing}
\qquad\=\hspace{\lwidth}\=\hspace{\cwidth}\=\+\kill
${\Conid{Subsum}\;\Varid{f}\;(\Conid{Sum}\;\Varid{f}\;\Varid{g})}$\\
${\Conid{Subsum}\;\Varid{f}\;\Varid{g}\Rightarrow \Conid{Subsum}\;\Varid{f}\;(\Conid{Sum}\;\Varid{x}\;\Varid{g})}$
\end{tabbing}
should always hold for any \ensuremath{\Varid{f}} and \ensuremath{\Varid{g}}.  The structure of the \ensuremath{\Conid{Sum}}
makes these definitions easier to write.

First, define \ensuremath{\Varid{f}} to be a \ensuremath{\Conid{Subsum}} of \ensuremath{\Conid{Sum}\;\Varid{f}\;\Varid{g}} by defining \ensuremath{\Varid{\uparrow}_{S}} and
\ensuremath{\Varid{\downarrow}_{S}} for all \ensuremath{\Varid{f}} and \ensuremath{\Varid{g}}:

\begin{tabbing}
\qquad\=\hspace{\lwidth}\=\hspace{\cwidth}\=\+\kill
${\mathbf{instance}\;\Conid{Subsum}\;\Varid{f}\;(\Conid{Sum}\;\Varid{f}\;\Varid{g})\;\mathbf{where}}$\\
${\hskip2.00em\relax\Varid{\uparrow}_{S}\mathrel{=}\Conid{S}\mathbin{\circ}\Conid{\Conid{Prelude}.Left}}$\\
${\hskip2.00em\relax\Varid{\downarrow}_{S}\;(\Conid{S}\;(\Conid{\Conid{Prelude}.Left}\;\Varid{f}))\mathrel{=}\Conid{Just}\;\Varid{f}}$\\
${\hskip2.00em\relax\Varid{\downarrow}_{S}\;(\Conid{S}\;(\Conid{\Conid{Prelude}.Right}\;\Varid{f}))\mathrel{=}\Conid{Nothing}}$
\end{tabbing}
Because \ensuremath{\Varid{f}} is always the left element of the \ensuremath{\Conid{Sum}}, \ensuremath{\Varid{\uparrow}_{S}} is defined
as the composition of the \ensuremath{\Conid{Sum}} constructor, \ensuremath{\Conid{S}}, and the \ensuremath{\Conid{Left}}
constructor from the built in \ensuremath{\Conid{Either}} type.  If \ensuremath{\Varid{f}} is the left
element of the \ensuremath{\Conid{Sum}}, then the projection function is simply \ensuremath{\Conid{Just}}.
If it is the right element of the sum, then the projection is
\ensuremath{\Conid{Nothing}} indicating that there is no project from the \ensuremath{\Conid{Sum}} back to
\ensuremath{\Varid{f}}.  This works specifically because the left element of the \ensuremath{\Conid{Sum}} is
always a leaf implying there is no need to descend a subtree created
by \ensuremath{\Conid{Sum}}.

Remember that \ensuremath{\Varid{f}} and \ensuremath{\Varid{g}} are type variables in the instance.  Thus,
\ensuremath{\Conid{Subsum}\;\Varid{f}\;(\Conid{Sum}\;\Varid{f}\;\Varid{g})} holds for all types \ensuremath{\Varid{f}} and \ensuremath{\Varid{g}}.  This is an
important result because we will get this \ensuremath{\Conid{Subsum}} for free each time
we define a \ensuremath{\Conid{Sum}}. The same can be said for other instances defined
over type variables rather than specific instances.

Second, if \ensuremath{\Conid{Subsum}\;\Varid{f}\;\Varid{g}} holds, then \ensuremath{\Conid{Subsum}\;\Varid{f}\;(\Conid{Sum}\;\Varid{x}\;\Varid{g})} also
holds. Unlike the first case, here the subtype relationship occurs in
the right element of the \ensuremath{\Conid{Sum}}.  Because the right element can be a
tree, we must descend the tree when creating the injection function.
Here \ensuremath{\Varid{\uparrow}_{S}} is virtually the same with the addition of a recursive
call to \ensuremath{\Varid{\uparrow}_{S}}.  This is an interesting function because it will
recursively call \ensuremath{\Varid{\uparrow}_{S}} until: (i) the injected element is in the type
associated with the left element of a sum; or (ii) the injected
element is the type associated with the right sum element of the
bottom subtree.

The projection function is similarly structured.  As long as the
function is looking at the right subtree, \ensuremath{\Varid{\downarrow}_{S}} is called recursively
until the type is found.  If the the left subtree is ever examined,
then \ensuremath{\Conid{Nothing}} is generated.  Note the lack of a case returning
\ensuremath{\Conid{Just}}. This will occur only projecting a subtype from itself and is
taken care of by an earlier definition.

\begin{tabbing}
\qquad\=\hspace{\lwidth}\=\hspace{\cwidth}\=\+\kill
${\mathbf{instance}\;(\Conid{Subsum}\;\Varid{f}\;\Varid{g})\Rightarrow \Conid{Subsum}\;\Varid{f}\;(\Conid{Sum}\;\Varid{x}\;\Varid{g})\;\mathbf{where}}$\\
${\hskip2.00em\relax\Varid{\uparrow}_{S}\mathrel{=}\Conid{S}\mathbin{\circ}\Conid{\Conid{Prelude}.Right}\mathbin{\circ}\Varid{\uparrow}_{S}}$\\
${\hskip2.00em\relax\Varid{\downarrow}_{S}\;(\Conid{S}\;(\Conid{\Conid{Prelude}.Left}\;\Varid{x}))\mathrel{=}\Conid{Nothing}}$\\
${\hskip2.00em\relax\Varid{\downarrow}_{S}\;(\Conid{S}\;(\Conid{\Conid{Prelude}.Right}\;\Varid{b}))\mathrel{=}\Varid{\downarrow}_{S}\;\Varid{b}}$
\end{tabbing}
Again, because \ensuremath{\Varid{f}} and \ensuremath{\Varid{g}} are type variables, this instance is
available for any two types that satisfy the instance constraints.
Specifically, if \ensuremath{\Conid{Subsum}\;\Varid{f}\;\Varid{g}} holds, we know that \ensuremath{\Conid{Subsum}\;\Varid{f}\;(\Conid{Sum}\;\Varid{x}\;\Varid{g})}
is defined and can be used.  This will prove valuable later when we
start defining languages.

The \ensuremath{\Varid{toSum}} function is a utility function that will inject a syntax
element into a \ensuremath{\Conid{Sum}}.  Specifically, if we have a \ensuremath{\Varid{f}} defined over the
fixed point of an expression language, \ensuremath{\Varid{g}}, we can automatically
inject members of \ensuremath{\Varid{f}} into the fixed point of \ensuremath{\Varid{g}} if \ensuremath{\Varid{f}} is a \ensuremath{\Conid{Subsum}}
of \ensuremath{\Varid{g}}.  Here \ensuremath{\Varid{f}} is the individual language component and \ensuremath{\Varid{g}} is the
\ensuremath{\Conid{Sum}} of a number of language components including \ensuremath{\Varid{f}}.  Thus, \ensuremath{\Varid{f}} is
a \ensuremath{\Conid{Subsum}} of \ensuremath{\Varid{g}} and an injection function exists.  This injection
function can then be used along with the \ensuremath{\Conid{Sum}} constructor to create
language elements.

\begin{tabbing}
\qquad\=\hspace{\lwidth}\=\hspace{\cwidth}\=\+\kill
${\Varid{toSum}\mathbin{::}(\Conid{Subsum}\;\Varid{f}\;\Varid{g},\Conid{Functor}\;\Varid{g})\Rightarrow \Varid{f}\;(\Conid{Rec}\;\Varid{g})\to \Conid{Rec}\;\Varid{g}}$\\
${\Varid{toSum}\;\Varid{x}\mathrel{=}\Conid{In}\;(\Varid{\uparrow}_{S}\;\Varid{x})}$
\end{tabbing}
\section{Example Language}

To demonstrate the use of these language definition features, we will
define an interpreter for an \ensuremath{\Conid{Integer}} language that implements simple
mathematical operations.  We start by defining types for each language
element:

\begin{tabbing}
\qquad\=\hspace{\lwidth}\=\hspace{\cwidth}\=\+\kill
${\mathbf{data}\;\Conid{ExprConst}\;\Varid{e}\mathrel{=}\Conid{EConst}\;\Conid{Int}}$\\
${\phantom{\mathbf{data}\;\Conid{ExprConst}\;\Varid{e}\mathrel{=}\mbox{}}\mathbf{deriving}\;(\Conid{Show},\Conid{Eq})}$\\
${}$\\
${\mathbf{instance}\;\Conid{Functor}\;\Conid{ExprConst}\;\mathbf{where}}$\\
${\hskip2.00em\relax\Varid{map}_{f}\;\Varid{f}\;(\Conid{EConst}\;\Varid{x})\mathrel{=}\Conid{EConst}\;\Varid{x}}$\\
${}$\\
${\mathbf{instance}\;\Conid{Algebra}\;\Conid{ExprConst}\;\Conid{Int}\;\mathbf{where}}$\\
${\hskip2.00em\relax\Varid{\phi}\;(\Conid{EConst}\;\Varid{x})\mathrel{=}\Varid{x}}$\\
${}$\\
${\mathbf{data}\;\Conid{ExprAdd}\;\Varid{e}\mathrel{=}\Conid{EAdd}\;\Varid{e}\;\Varid{e}}$\\
${\phantom{\mathbf{data}\;\Conid{ExprAdd}\;\Varid{e}\mathrel{=}\mbox{}}\mathbf{deriving}\;(\Conid{Show},\Conid{Eq})}$\\
${}$\\
${\mathbf{instance}\;\Conid{Functor}\;\Conid{ExprAdd}\;\mathbf{where}}$\\
${\hskip2.00em\relax\Varid{map}_{f}\;\Varid{f}\;(\Conid{EAdd}\;\Varid{x}\;\Varid{y})\mathrel{=}(\Conid{EAdd}\;(\Varid{f}\;\Varid{x})\;(\Varid{f}\;\Varid{y}))}$\\
${}$\\
${\mathbf{instance}\;\Conid{Algebra}\;\Conid{ExprAdd}\;\Conid{Int}\;\mathbf{where}}$\\
${\hskip2.00em\relax\Varid{\phi}\;(\Conid{EAdd}\;\Varid{x}\;\Varid{y})\mathrel{=}\Varid{x}\mathbin{+}\Varid{y}}$\\
${}$\\
${\mathbf{data}\;\Conid{ExprMult}\;\Varid{e}\mathrel{=}\Conid{EMult}\;\Varid{e}\;\Varid{e}}$\\
${\phantom{\mathbf{data}\;\Conid{ExprMult}\;\Varid{e}\mathrel{=}\mbox{}}\mathbf{deriving}\;(\Conid{Show},\Conid{Eq})}$\\
${}$\\
${\mathbf{instance}\;\Conid{Functor}\;\Conid{ExprMult}\;\mathbf{where}}$\\
${\hskip2.00em\relax\Varid{map}_{f}\;\Varid{f}\;(\Conid{EMult}\;\Varid{x}\;\Varid{y})\mathrel{=}(\Conid{EMult}\;(\Varid{f}\;\Varid{x})\;(\Varid{f}\;\Varid{y}))}$\\
${}$\\
${\mathbf{instance}\;\Conid{Algebra}\;\Conid{ExprMult}\;\Conid{Int}\;\mathbf{where}}$\\
${\hskip2.00em\relax\Varid{\phi}\;(\Conid{EMult}\;\Varid{x}\;\Varid{y})\mathrel{=}\Varid{x}\mathbin{*}\Varid{y}}$
\end{tabbing}
We avoided defining expressions recursively by pulling the recursive
instance out and making it a parameter, called \ensuremath{\Varid{e}}, in all the
expression types.  Thus, it is not possible to define an \ensuremath{\Conid{ExprAdd}}
over another \ensuremath{\Conid{ExprAdd}} because we don't know what \ensuremath{\Varid{e}} is.  We can use
the \ensuremath{\Conid{Rec}} type constructor to define the fixed point for each language
element.  For example, if we evaluate the following definition:

\begin{tabbing}
\qquad\=\hspace{\lwidth}\=\hspace{\cwidth}\=\+\kill
${\Conid{ExprAddLang}\mathrel{=}\Conid{Rec}\;\Conid{ExprAdd}}$
\end{tabbing}
\ensuremath{\Conid{ExprAddLang}} is the fixed point of ExprAdd.  Every element of
\ensuremath{\Conid{ExprAddLang}} is constructed of \ensuremath{\Conid{ExprAdd}} defined over other \ensuremath{\Conid{ExprAdd}}
expressions.  This is not particularly useful in itself because it
does not include other language elements, only \ensuremath{\Conid{ExprAdd}}.  Thus, we
will not include these definitions.

Now we combine the individual types into a single type using \ensuremath{\Conid{Sum}}:

\begin{tabbing}
\qquad\=\hspace{\lwidth}\=\hspace{\cwidth}\=\+\kill
${\mathbf{type}\;\Conid{ExprVal}\mathrel{=}(\Conid{Sum}\;\Conid{ExprConst}}$\\
${\phantom{\mathbf{type}\;\Conid{ExprVal}\mathrel{=}(\mbox{}}(\Conid{Sum}\;\Conid{ExprAdd}\;\Conid{ExprMult}))}$
\end{tabbing}
The \ensuremath{\Conid{ExprVal}} type is still parameterized over the expression type
\ensuremath{\Varid{e}}.  Using \ensuremath{\Conid{ExprVal}} and \ensuremath{\Varid{toSum}} we can instantiate any of the term
types as elements of \ensuremath{\Conid{ExprVal}}.  However, we still cannot define terms
over other terms because the parameter \ensuremath{\Varid{e}} remains unspecified.  What
we would like is for \ensuremath{\Varid{e}} the same set of terms that comprise
\ensuremath{\Conid{ExprVal}}.  This is exactly what the \ensuremath{\Conid{Rec}} constructor for fixed point
types provides:

\begin{tabbing}
\qquad\=\hspace{\lwidth}\=\hspace{\cwidth}\=\+\kill
${\mathbf{type}\;\Conid{ExprLang}\mathrel{=}\Conid{Rec}\;\Conid{ExprVal}}$
\end{tabbing}
The type \ensuremath{\Conid{ExprLang}} defines the complete expression language as
\ensuremath{\Conid{ExprVal}}, the constructor for expressions, instantiated with itself.
Said differently, \ensuremath{\Conid{ExprLang}} is the collection of expressions defined
over expressions.

Isn't \ensuremath{\Conid{ExprLang}} infinite?  It certainly should be because an infinite
number of terms can be defined in our expression language.  If that is
the case, shouldn't the definition of \ensuremath{\Conid{ExprLang}} be nonterminating?
Lazy evaluation helps here.  Remember that elements of the collection
of terms are not calculated until they are needed.

We will add a convenience function, \ensuremath{\Varid{toExprLang}}, that will take any
of the language elements and project it into the full expression
language.  The function signature here is important as it adds a
constraint to the application.  Specifically, if \ensuremath{\Varid{f}} is a \ensuremath{\Conid{Subsum}} of
\ensuremath{\Conid{ExprVal}}, then \ensuremath{\Varid{f}} over \ensuremath{\Conid{ExprLang}} can be mapped to \ensuremath{\Conid{ExprLang}} using
\ensuremath{\Varid{toSum}}.  What the signature does is constrain the types being
manipulated by \ensuremath{\Varid{toSum}} and establish the existence of an injection
function between \ensuremath{\Varid{f}} and \ensuremath{\Conid{ExprVal}}.

We know that if \ensuremath{\Varid{f}} is one of the expression elements, it is a
\ensuremath{\Conid{Subsum}} of \ensuremath{\Conid{ExprVal}} because \ensuremath{\Conid{ExprVal}} is created with a \ensuremath{\Conid{Sum}}.  This
works like the inference that the \ensuremath{\Conid{Sum}} is a \ensuremath{\Conid{Functor}} and \ensuremath{\Conid{Algebra}}.
Now we have \ensuremath{\Varid{\downarrow}} defined between \ensuremath{\Varid{f}} and \ensuremath{\Conid{ExprVal}} as defined by the
earlier type class.  With that, \ensuremath{\Varid{toSum}} can do it's job over elements
of \ensuremath{\Conid{ExprLang}}.

\begin{tabbing}
\qquad\=\hspace{\lwidth}\=\hspace{\cwidth}\=\+\kill
${\Varid{toExprLang}\mathbin{::}(\Conid{Subsum}\;\Varid{f}\;\Conid{ExprVal})\Rightarrow \Varid{f}\;\Conid{ExprLang}\to \Conid{ExprLang}}$\\
${\Varid{toExprLang}\mathrel{=}\Varid{toSum}}$
\end{tabbing}
We like to define have \ensuremath{\Conid{ExprVal}} to be a \ensuremath{\Conid{Functor}} and an \ensuremath{\Conid{Algebra}}
giving us \ensuremath{\Varid{map}_{f}} and \ensuremath{\Varid{\phi}} over the entire expression language.  As it
turns out, we get this for free from existing definitions.  In the
definitions associated with \ensuremath{\Conid{Sum}} of language element, we provided two
definitions repeated here:

\begin{tabbing}
\qquad\=\hspace{\lwidth}\=\hspace{\cwidth}\=\+\kill
${\mathbf{instance}\;(\Conid{Functor}\;\Varid{f},\Conid{Functor}\;\Varid{g})\Rightarrow \Conid{Functor}\;(\Conid{Sum}\;\Varid{f}\;\Varid{g})\;\mathbf{where}}$\\
${\hskip2.00em\relax\Varid{map}_{f}\;\Varid{h}\;(\Conid{S}\;(\Conid{\Conid{Prelude}.Left}\;\Varid{x}))\mathrel{=}\Conid{S}\;(\Conid{\Conid{Prelude}.Left}\;(\Varid{map}_{f}\;\Varid{h}\;\Varid{x}))}$\\
${\hskip2.00em\relax\Varid{map}_{f}\;\Varid{h}\;(\Conid{S}\;(\Conid{\Conid{Prelude}.Right}\;\Varid{x}))\mathrel{=}\Conid{S}\;(\Conid{\Conid{Prelude}.Right}\;(\Varid{map}_{f}\;\Varid{h}\;\Varid{x}))}$\\
${}$\\
${\mathbf{instance}\;(\Conid{Algebra}\;\Varid{f}\;\Varid{a},\Conid{Algebra}\;\Varid{g}\;\Varid{a})\Rightarrow \Conid{Algebra}\;(\Conid{Sum}\;\Varid{f}\;\Varid{g})\;\Varid{a}}$\\
${\hskip2.00em\relax\mathbf{where}\;\Varid{\phi}\mathrel{=}\Varid{either}\;\Varid{\phi}\;\Varid{\phi}\mathbin{\circ}\Varid{unS}}$
\end{tabbing}
Both definitions work in the same way.  In the first, if \ensuremath{\Varid{f}} and \ensuremath{\Varid{g}}
are instances of \ensuremath{\Conid{Functor}}, then \ensuremath{\Conid{Sum}\;\Varid{f}\;\Varid{g}} is also an instance of
\ensuremath{\Conid{Functor}} with \ensuremath{\Varid{map}_{f}} defined as shown.  Because all of the data types
combined by the \ensuremath{\Conid{Sum}} are instances of \ensuremath{\Conid{Functor}}, so is the \ensuremath{\Conid{Sum}}.
This comes for free from the definition.  The same is true for
\ensuremath{\Conid{Algebra}}.  As long as new elements added to \ensuremath{\Conid{ExprLangVal}} are
instances of \ensuremath{\Conid{Functor}} and \ensuremath{\Conid{Algebra}}, \ensuremath{\Conid{ExprLangVal}} will continue to
be an instance of \ensuremath{\Conid{Functor}} and \ensuremath{\Conid{Algebra}}.  This is a rather amazing
result that will save substantial effort as we add to our language.

The easiest way to define an eval function for the language is to use
the \ensuremath{\Varid{cata}} function.  We'll include a type signature to help make
types of return values work out:

\begin{tabbing}
\qquad\=\hspace{\lwidth}\=\hspace{\cwidth}\=\+\kill
${\Varid{eval0}\mathbin{::}\Conid{ExprLang}\to \Conid{Int}}$\\
${\Varid{eval0}\mathrel{=}\Varid{cata}}$
\end{tabbing}
The penalty for defining languages in this way is a significantly more
cryptic and complex abstract syntax definition.  Using the abstract
syntax directly requires projecting language elements into the main
language before evaluation.  Following are some specific examples that
show how the \ensuremath{\Varid{toExprLang}} utility function is used to experiment with
different abstract syntax constructs:

\begin{tabbing}
\qquad\=\hspace{\lwidth}\=\hspace{\cwidth}\=\+\kill
${\Varid{test}_{0}\mathbin{::}\Conid{ExprLang}\mathrel{=}\Varid{toExprLang}\;(\Conid{EConst}\;\mathrm{1})}$\\
${\Varid{test}_{1}\mathbin{::}\Conid{ExprLang}\mathrel{=}\Varid{toExprLang}\;(\Conid{EAdd}}$\\
${\phantom{\Varid{test}_{1}\mathbin{::}\Conid{ExprLang}\mathrel{=}\Varid{toExprLang}\;(\mbox{}}(\Varid{toExprLang}\;(\Conid{EConst}\;\mathrm{1}))}$\\
${\phantom{\Varid{test}_{1}\mathbin{::}\Conid{ExprLang}\mathrel{=}\Varid{toExprLang}\;(\mbox{}}(\Varid{toExprLang}\;(\Conid{EConst}\;\mathrm{2})))}$\\
${\Varid{test}_{2}\mathbin{::}\Conid{ExprLang}\mathrel{=}\Varid{toExprLang}\;(\Conid{EAdd}}$\\
${\phantom{\Varid{test}_{2}\mathbin{::}\Conid{ExprLang}\mathrel{=}\Varid{toExprLang}\;(\mbox{}}(\Varid{toExprLang}\;(\Conid{EMult}}$\\
${\phantom{\Varid{test}_{2}\mathbin{::}\Conid{ExprLang}\mathrel{=}\Varid{toExprLang}\;(\mbox{}}\phantom{(\Varid{toExprLang}\;(\mbox{}}(\Varid{toExprLang}\;(\Conid{EConst}\;\mathrm{2}))}$\\
${\phantom{\Varid{test}_{2}\mathbin{::}\Conid{ExprLang}\mathrel{=}\Varid{toExprLang}\;(\mbox{}}\phantom{(\Varid{toExprLang}\;(\mbox{}}(\Varid{toExprLang}\;(\Conid{EConst}\;\mathrm{4}))))}$\\
${\phantom{\Varid{test}_{2}\mathbin{::}\Conid{ExprLang}\mathrel{=}\Varid{toExprLang}\;(\mbox{}}(\Varid{toExprLang}\;(\Conid{EConst}\;\mathrm{1})))}$
\end{tabbing}
\section{Conclusions}

So why would anyone ever define an interpreter this way?  Very simply
because it is trivial to add new elements to the language.  This
involves a three step process of: (i) defining the new language
element; (ii) making the element an instance of \ensuremath{\Conid{Functor}} and
\ensuremath{\Conid{Algebra}}; and (iii) adding the new language element to the \ensuremath{\Conid{Sum}}
defining the language. There is no need to touch the existing language
elements or the \ensuremath{\Varid{eval}} function.

\end{document}
